\chapter{Diskussion}
\label{cha:Diskussion}
Zu Beginn wurde die Intensitätsverteilung des Röntgenstrahls untersucht. Die Intensitätsverteilung stimmt ausreichend gut mit einer Gaußverteilung überein. Dies war zu erwarten und 
ist ein Indiz für eine qualitative Detektorposition. Unterstützt wird dies durch die sehr geringe Konstante im Fit. 

Als nächstes wurde die Strahlbreite untersucht. Der Z-Scan lieferte eine Strahlbreite von $d_0 \approx \qty{0.36}{\milli\metre}$. Allerdings ist in den Messdaten die Abrundung bei 
dem Abfall der Intensität nahe Null sehr gering. Daher ist die Strahlbreite mit einer gewissen Unsicherheit behaftet. Aus der Strahlbreite konnte ein Geometriewinkel von 
$\theta_\text{g, Theorie} = \qty{1.03}{\degree}$ bestimmen.

Danach wurde der Geometriewinkel zur Korrektur der Reflektivitäten bestimmt. Dieser lautet $\theta_\text{g} = \qty{0.96}{\degree}$. Die relative Abweichung der beiden Winkel 
aus den unterschiedlichen Methoden liegt bei $\qty{6.7}{\percent}$. Die Abweichung zwischen den beiden Methoden ist relativ gering und liegt im Bereich einer statistischen Unsicherheit.

Weiter wurde die Reflektivität gemessen. Mit einer Korrektur durch einen Diffuse-Scan ergeben sich Messdaten, welche die erwarteten Kiessing-Oszillationen zeigen. Leider sind diese 
Oszillationen sehr schwach, was zu Ungenauigkeiten in der Bestimmung der Minima führt. Der kritische Winkel ist in den Messwerten gut erkennbar. Die aus den 
Kiessing-Oszillationen bestimmte Schichtdicke ergibt sich zu $d = \qty{8.46 +- 0.73e-8}{\metre}$.

Zum Vergleich wurde die Reflektivität durch den Parrattalgorithmus gefittet. Aufgrund der vielen Parameter konnte leider kein guter Fit gefunden werden. Dementsprechend 
sind die damit bestimmten Werte auch durch große Unsicherheiten behaftet. \\
Die bestimmten Parameter für die Dispersion sind $\delta_\text{Poly} = \num{0.61e-6}$ und 
$\delta_\text{Si} = \num{7.32e-6}$. Die Literaturwerte der Dispersion lauten $\delta_\text{Poly, lit} = \num{3.5e-6}$ und $\delta_\text{Si, lit} = \num{7.6e-6}$ \cite{V44}.
Dies sind relative Abweichungen von $\qty{82.6}{\percent}$ und $\qty{3.7}{\percent}$. \\
Aus der Dispersion konnten die kritischen Winkel 
$\theta_\text{c, Poly} = \qty{0.0633}{\degree}$ und $\theta_\text{c, Si} = \qty{0.2192}{\degree}$ berechnet werden. Die Literaturwerte der kritischen Winkel lauten 
$\theta_\text{c, lit}(\text{Poly}) =\qty{0.153}{\degree}$ und $\theta_\text{c, lit}(\text{Si}) =\qty{0.223}{\degree}$. Die experimentellen Ergebnisse weisen relative 
Abweichungen von $\qty{58.6}{\percent}$ für Polysterol und $\qty{1.7}{\percent}$ für Silizium. \\
Ebenfalls konnte hieraus die Schichtdicke zu $d = \qty{8.46e-8}{\metre}$
bestimmt werden. Im Rahmen der Messunsicherheit bestätigen die beiden Werte sich. Erst hohe Nachkommastellen weisen Abweichungen auf.
Zuletzt wurde die Rauigkeit bestimmt. Dafür ergibt sich $\sigma_\text{Luft, Poly} = \num{7.90e-10}$ und $\sigma_\text{Poly, Si} = \num{7.30e-10}$.

Aufgrund des ungenauen Fits des Parrattalgorithmus konnten die Parameter nicht mit genügender Sicherheit bestimmt werden. 

Zusammengefasst sind die experimentellen Ergebnisse nicht qualitativ. Obgleich das Ziel des Versuches, die Auseinandersetzung mit Oberflächenuntersuchungen und das Verständnis 
für Vielschichtensysteme, mit Erfolg erlangt wurde. 