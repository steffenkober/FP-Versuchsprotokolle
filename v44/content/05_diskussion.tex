\chapter{Diskussion}
\label{cha:Diskussion}
Zu Beginn wurde die Intensitätsverteilung des Röntgenstrahls untersucht. Die Intensitätsverteilung stimmt relativ genau mit einer Gaußverteilung überein. Dies war zu erwarten und 
ist ein Indiz für eine Qualitative Detektorposition. Unterstützt wird dies durch die sehr geringe Konstante im Fit. 

Als nächstes wurde die Strahlbreite untersucht. Der Z-Scan lieferte eine Strahlbreite von $d_0 \approx \qty{0.36}{\milli\metre}$. Allerdings ist in den Messdaten die Abrundung bei 
dem Abfall der Intensität nahe Null sehr gering. Daher ist die Strahlbreite mit eine gewissen Unsicherheit behaftet. Aus der Strahlbreite konnte ein Geometriewinkel von 
$\alpha_\text{G, Theorie} = \qty{1.03}{\degree}$ bestimmen.

Danach wurde der Geometriewinkel zur Korrektur der Reflektivitäten bestimmt. Dieser wurde zu $\alpha_\text{G} = \qty{0.96}{\degree}$ bestimmt. Die relative Abweichung der beiden Winkel 
aus den unterschiedlichen Methoden liegt bei $\qty{6.7}{\percent}$. Die Abweichung zwischen den beiden Methode ist relativ gering und liegt im Bereich eine statistischen Unsicherheit.
