\chapter{Auswertung}
\label{cha:Auswertung}
% hier noch unsere fehlerdaten einbauen
Als erstes muss das Setup justiert werden. Dies ist nötig um die genaue Position der Probe zu finden, sowie den Geometriefaktor zu bestimmen mit Hilfe welchen die Messdaten 
hinterher korrigiert werden müssen.

\section{Bestimmung der Halbwertsbreite und der maximalen Intensität}
\label{sec:FWHM}
Das Intensitätsprofil der Röntgenstrahlen wird ohne Probe im Strahlengang vermessen. Dazu wird der Detektor relativ zum Emitter um kleine Winkel gedreht. Die damit aufgenommenen 
Messdaten folgen einer Gaußverteilung. Die in Abbildung \ref{fig:DScan} dargestellten Messdaten werden an die Gaußfunktion 
\begin{equation*}
    I(\alpha) = \frac{I_0}{\sqrt{2\symup{\pi}\sigma^2}}\cdot \symup{exp}\left(-\frac{(\alpha - \alpha_0)^2}{2\sigma^2}\right) + C
\end{equation*}
mittels Scipy \cite{scipy} gefittet.

\begin{figure}
    \centering
    \includegraphics[width = .7\textwidth]{DScan.pdf}
    \caption{Messdaten des Detektorscans, sowie ein Gaußfit und die Bestimmung der Halbwärtsbreite sind in diesem Graphen dargestellt.}
    \label{fig:DScan}
  \end{figure}

Die Parameter des Fits lauten
\begin{align*}
    \alpha_0 &= \qty{4.95 +-0.72e-3}{\degree} \\
    \sigma &= \qty{6.86 +- 0.08e-2}{\degree} \\
    I_0 &= \num{4.88 +- 0.06e5} \\
    C &= \num{6.40 +- 9.07e3}.
\end{align*}

Die Halbwärtsbreite (FWHM) gibt die Breite der Verteilung auf halber Intensität an. Für eine Gaußverteilung gilt $\text{FWHM} = 2\sqrt{2\ln\left(2\right)}\sigma$. Durch die 
Fitparameter ergibt sich 
\begin{equation*}
    \text{FWHM} = \qty{0.1616+-0.0018}{\degree}
\end{equation*}

\section{Bestimmung der Strahlbreite}
\label{sec:ZScan}
Zur Bestimmung der Strahlenbreite wird ein Z-Scan durchgeführt. Bevor die Probe im Strahlengang liegt sollte die volle Intensität vorhanden sein. Bewegt sich der Strahl dann aber in 
die Probe wird er stetig abgeschwächt bis er vollkommen geblockt wird. Die Breite dieses Abfalls bestimmt daher die Strahlenbreite. Die Messdaten des Z-Scans sind in Abbildung 
\ref{fig:ZScan} dargestellt.

\begin{figure}
    \centering
    \includegraphics[width = .7\textwidth]{ZScan.pdf}
    \caption{Messdaten des Z-Scans, welcher ein Abfallintervall aufweist. Dieses Intervall entspricht der Strahlenbreite.}
    \label{fig:ZScan}
\end{figure}

Aus dem Plot ergibt sich eine Strahlbreite von $d_0 \approx \qty{0.36}{\milli\metre}$. 

\section{Bestimmung des Geometriewinkels}
\label{sec:Rocking}

Die soeben bestimmte Strahlbreite muss in der Analyse berücksichtigt werden. Trifft der Röntgenstrahl unter eine Winkel auf die Probe, so ist die gesamte Breite des Strahls größer als 
die Probe selbst. Somit kann nicht der gesamte Strahl reflektiert werden und es resultiert eine verminderte gemessene Intensität. Um dieses Problem zu lösen wird ein Geometriewinkel 
bestimmt, bis zu welchem dieses Problem besteht. Dieser Geometriewinkel wird durch einen Rocking-Scan bestimmt. Die aufgenommene Intensitätsverteilung ist in Abbildung 
\ref{fig:RockingScan} dargestellt. 
\begin{figure}
    \centering
    \includegraphics[width = .7\textwidth]{RockingScan.pdf}
    \caption{Rocking-Scan zur Bestimmung des Geometriewinkels.}
    \label{fig:RockingScan}
\end{figure}
Der Geometriewinkel entspricht dann der Winkelbreite in welcher die Intensität vom Maximum bis auf Null abfällt. Daher lautet der Geometriewinkel für diese Justage
$\alpha_\text{G} = \qty{0.96}{\degree}$. Aus der strahlbreite lässt sich gemäß Gleichung \ref{eqn:Geometriewinkel} der theoretische Geometriewinkel zu 
$\alpha_\text{G, Theorie} = \qty{1.03}{\degree}$ bestimmen.

\section{Bestimmung der Dispersion und Rauigkeit des Siliziumwafers}
\label{sec:dis_rau}
Als nächstes wird die Dispersion und Rauigkeit des Siliziumwafers bestimmt. Dazu wird eine Messung der Reflektivität, sowie eine Referenzmessung zur Bestimmung der gestreuten
Intensität, durchgeführt. Die Messung der Reflektivität wird um die Referenzmessung korrigiert. Die Messdaten, sowie die korrigierten Daten sind in Abbildung \ref{fig:Reflek1}
dargestellt.
\begin{figure}
    \centering
    \includegraphics[width = .7\textwidth]{Reflek1.pdf}
    \caption{Reflectivity-Scan und Diffuse-Scan, sowie der korrigerte Reflectivity-Scan.}
    \label{fig:Reflek1}
\end{figure}

Für einen adäquaten Vergleich zu theoretischen Vorhersagen wird die Intensität des Reflektivitäts-Scans in eine richtige Reflektivität umgerechnet. Dies erfolgt durch ein 
Messintervall von $\qty{5}{\second}$ mit 
\begin{equation*}
    R = \frac{I}{5I_0}.
\end{equation*}
Nun wird die Korrektur um den Geometriefaktor einbezogen, welche sich nach Gleichung \ref{eqn:Geometriefaktor} richtet. Das einfachste Modell zur Beschreibung der 
Reflektivität ist durch die Fresnelreflektivität \ref{eqn:fresnel_ref} gegeben. Dieses Modell berücksichtigt noch keine Rauigkeit. Der kritisches Winkel ab welchem ein Teil des
Strahls transmittiert wird ist durch Gleichung \ref{eqn:alpha_total} gegeben. Mit der Dichte multipliziert mit dem klassischen Elektronenradius $r_e\rho = \qty{20e14}{\metre^{-2}}$ \cite{V44} und der Wellenlänge 
der $K_\alpha$-Linie ergibt sich $\alpha_\text{c} = \qty{0.223}{\degree}$. Die korrigierte Reflektivität, sowie die Fresnelreflektivität sind in Abbildung \ref{fig:Reflek2}
dargestellt.

\begin{figure}
    \centering
    \includegraphics[width = .7\textwidth]{Reflek2.pdf}
    \caption{Die korrigierten Messdaten des Reflektivität-Scans sowie die Fresnelreflektivität einer 
    ideal glatten Polysterol Oberfläche sind in dieser Abbildung dargestellt. Ebenso ist der kritische Winkel und die Kiessing-Oszillationen eingezeichnet.}
    \label{fig:Reflek2}
\end{figure}

Die Schichtdicke des Polysterolfilms kann anhand der auftretenden Kiessing-Oszillationen bestimmt werden. Dazu werden die Minima der Oszillationen in Abbildung \ref{fig:Reflek2}
markiert. Aus dem mittleren Abstand der Minima $\symup{\Delta}\alpha = \qty{5.21 +- 0.45e-2}{\degree}$ kann gemäß Gleichung \ref{eqn:Schichtdicke} die Schichtdicke zu 
$d = \qty{8.46 +- 0.73e-8}{\metre}$ bestimmt werden.
Durch den Parrattalgorithmus kann die Reflektivität eines Vielschichtensystems genauer bestimmt werden. Für noch bessere Ergebnisse wird der Fresnelkoeffizient noch durch eine 
Rauigkeit gemäß Gleichung \ref{eqn:rauigkeit} korrigert. Der verwendete Parrattalgorithmus findet sich im Anhang \ref{anh:parratt}. Dieses Modell wird dann an die korrigierten
Messdaten mittels Scipy \cite{scipy} gefittet. Aufgrund der vielen Variablen wurde Start- und Grenzewerte gewählt um im Rahmen Möglichkeiten eine Konvergenz der Optimierung 
zu erreichen. Die gefittete Funktion, sowie die Messdaten werden in Abbildung \ref{fig:Reflek3} dargestellt.
\begin{figure}
    \centering
    \includegraphics[width = .7\textwidth]{Reflek3.pdf}
    \caption{Messdaten der Reflektivität und der ermittelte Fit durch den Parrattalgorithmus unter Berücksichtigung der Rauigkeit.}
    \label{fig:Reflek3}
\end{figure}
Aus dem Fit folgen die Paremter zur Dispersion $\delta$, der Rauigkeit $\sigma$ und der Schichtdicke $d$
\begin{align*}
    &\delta_\text{Poly} = \num{3.0e-6} & &\delta_\text{Si} = \num{1.5e-5} \\
    &\beta_\text{Poly} = \num{5.6e-8} & &\beta_\text{Si} = \num{4.4e-7} \\
    &\sigma_\text{Luft, Poly} = \num{6.4e-10} & &\sigma_\text{Poly, Si} = \num{2.0e-10} \\
    &d = \qty{8.46e-8}{\metre}.
  \end{align*}

Aus Gleichung \ref{eqn:alpha_tot} können die kritischen Winkel der beiden Schichten berechnet werden. Diese lauten $\alpha_\text{c, Poly} = \qty{0.32}{\degree}$ und $\alpha_\text{c, Si} = \qty{0.14}{\degree}$.