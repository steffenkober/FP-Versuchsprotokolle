\chapter{Diskussion}
\label{cha:Diskussion}

\section{Energiekalibration}

Die Analyse des Europium-152 Spektrums ergab einen Zusammenhang zwischen Kanalnummer und Energiewert zu
\begin{equation}
    \mathrm{E(Kanal)} = \qty{0.2075 \pm 0.0001}{} \cdot \mathrm{Kanal} + \qty{-1.259 \pm 0.219}{}.
\end{equation}
Die geringe Unsicherheit des ersten Koeffizienten spricht für eine hohe Genauigkeit. Der zweite Koeffizient hat allerdings
eine Abweichung von circa $\qty{17}{\%}$ zum Koeffizienten selber. Dies und der Fakt, dass der Koeffizient negativ ist, deutet
auf einen systematischen Fehler des Detektors hin. Es wäre nämlich zu erwarten, dass Kanalnummer $0$ einer Energie von $\qty{0}{\kilo\electronvolt}$
oder zumindest einem positiven Energiewert zuzuordnen ist. Durch den negativen Faktor würde hier allerdings ein negativer Energiewert
resultieren, was unphysikalisch ist. Da der Parameter nichtsdestotrotz sehr gering ist, ist diese Fehlerquelle zu vernachlässigen.

\section{Vollenergienachweiswahrscheinlichkeit}

Weitergehend wurde durch die Analyse des Europium-152 eine Funktion für die Vollenergienachweiswahrscheinlichkeit in Abhängigkeit der Energie gefunden:
\begin{equation}
    Q(E) = \qty{110.6 \pm 41.3}{}\cdot E^{\qty{-1.053 \pm 0.064}{}} \qty{}{\per\kilo\electronvolt}.
\end{equation}
Auffällig ist hier die große Standardabweichung von etwa $\qty{37}{\%}$ des ersten Parameters. Grund hierfür muss eine fehlerhafte Berechnung des 
Vollenergienachweiswahrscheinlichkeit sein oder der systematische Fehler, welcher bereits bei der Bestimmung des Energie-Kanal-Zusammenhangs auftrat.
Die Vollenergienachweiswahrscheinlichkeit könnte aufgrund ungenauer Berechnung des Raumwinkels oder der Bestimmung des Linieninhalts resultieren.
Die hohe Unsicherheit des ersten Parameters wird an in der Analyse folgende Berechnungen vererbt werden, wodurch hohe Ungenauigkeiten entstehen.

\section{Monochromatisches Gammaspektrum eines Caesium-137-Strahlers}

Ein Vergleich der Quotienten der Halbwerts- und Zehntelwertsbreite einer Normalverteilung und der Messwerte zeigt, dass die Messwerte sehr gut mit einer
Normalverteilung beschrieben werden. Hier liegt erst in der sechsten Nachkommastelle eine Abweichung vor.\\
Die berechneten Parameter der Normalverteilung zeigen ebenfalls geringe Abweichungen auf:
\begin{align}
    a &= \qty{10370 \pm 40}{},\\
    \mu &= \qty{3193 \pm 0}{},\\
    \sigma &= \qty{4.740 \pm 0.019}{}.
\end{align}
Bei der Bestimmung der Ausgleichsfunktion des Compton-Kontinuums tritt ebenfalls eine geringe Abweichung des Parameters auf:
\begin{equation}
    a = \qty{9.205 \pm 0.080}{}.
\end{equation}
Dadurch, dass nur ein Parameter gefittet wird, ist die Funktion zudem extra genau.\\
Durch diese sehr genau bestimmten Funktionen kann der Linieninhalt des FEP und des Compton-Kontinuums mit einer Ungenauigkeit von kleiner als $\qty{1e-8}{}$
bestimm werden. Ein Vergleich dieser Linieninhalte zeigt eine Proportionlität des Inhalts des Compton-Kontinuums von $\qty{3.53}{}$ gegenüber dem Photopeak. Anhand der Berechnung der Absorptionswahrscheinlichkeit
würde allerdings eine Proportionlität von $\qty{26}{}$ erwartet werden. Diese Abweichung ist extrem hoch und muss durch Effekte entstehen, welche in dieser Analyse nicht berücksichtigt wurden.
Mögliche Gründe wären hier eine ungenaue Bestimmung des Linieninhalts des Compton-Kontinuums aufgrund der Störeffekte im Niederenergiebereich.

\section{Aktivitätsbestimmung von Barium-133}

Die Linieninhalte der FEP konnten erneut mit hoher Genauigkeit bestimmt werden. Auffällig sind hier lediglich die hohe Ungenauigkeit der Aktivitätsbestimmung. Diese ist 
ein Folgeeffekt der ungenauen Parameterbestimmung bei der Ausgleichsfunktion der Vollenergienachweiswahrscheinlichkeit. Eine Beseitigung des oben genannten systematischen 
Fehlers könnte die Genauigkeit hier um ein Vielfaches erhöhen.

\section{Nuklididentifizierung und Aktivitätsbestimmung von Uranophan}

Durch eine Analyse des Gammaspektrums des Uranophan ließen sich die Nuklide Radium-226, Blei-214 und Bismut-214 ausfindig machen, welche allesamt zur natürlichen Zerfallskette
des Uran zu Radium gehören. Die hohen Unsicherheiten der Aktivität sind erneut auf die ungenaue Bestimmung des Parameters bei der Vollenergienachweiswahrscheinlichkeit zurückzuführen.