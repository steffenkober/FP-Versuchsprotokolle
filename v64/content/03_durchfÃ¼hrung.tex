\chapter{Durchführung}
\label{cha:Durchführung}
Für die Druchführung dieses Versuches wird ein \textit{Sagnac-Interferometer} verwendet. Der Versuchsaufbau ist \autoref{fig:Aufbau} zu entnehmen. In dieser Abbildung \ref{fig:Aufbau}
ist grundlegend das Sagnac-Interferometer dargestellt, sowie die Laserquelle. Bei der Laserquelle handelt es sich um einen $\ce{HeNe}$-Laser mit der Wellenlänge $\lambda = \qty{632.990}{\nano\metre}$\cite{v64}.
Außerdem ist in \autoref{fig:Aufbau} eine Translations-Stage beim Spiegel \textit{M2} eingezeichnet. Anhand dieser Stage sollen später die Strahlengänge getrennt werden. Nach dem
Spiegel \textit{M2} trifft der Laserstrahl auf einen Strahlteilerwürfel, kurz \textit{PBSC}. Die Funktionsweise eines Strahlteilerwürfels wurde bereits in \autoref{sec:Sagnac} erklärt.
Nachdem der Strahl nun aufgeteilt wurde durchlaufen die Strahlen zunächst die Spiegel $\textit{M}_A$, $\textit{M}_B$ und $\textit{M}_C$ jeweils in entgegengesetzter Richtung. 
Danach treffen beide Strahlen im selben Punkt erneut auf dem \textit{PBSC} wodurch beide Strahlen dann in Richtung \textit{Output} reflektiert werden. Dabei verlaufen die Strahlen 
identisch und können somit detektiert werden. Der Detektor besteht selbst zunächst aus einem weiteren \textit{PBSC}. Der transmittierte Strahl läuft dann sofort in eine Photodiode,
wohingegen der andere Strahl ein kurzes Stück versetzt über einen Spiegel in eine andere Photodiode geleitet wird. Diese Aufspaltung des Detektors liefert die Grundlage für die 
Differenzspannungsmethode. Nun kann mit der Justage begonnen werden. Theoretisch müsste der gesamte Versuchsaufbau nun in $\qty{45}{\degree}$ aufgabaut werden. Da allerdings schon
kleine Schwankungen für relativ große Änderungen im Strahlengang sorgen muss man die Spiegel durch Justageschrauben noch passend ausrichten, damit die Strahlen wirklich in einem 
Punkt enden. Zu Beginn der Justage wird zunächst ein Polarisationsfilter hinter den Laser gestellt, damit wir direkt einen um $\qty{45}{\degree}$ zur vertikalen gedreht polarisierten
Strahl haben.  Ein zweiter Polarisationsfilter wird dann vor den Detektor eingebaut damit die Interferenzeffekte, anhand dessen wir das Experiment gestalten, überhaupt erst auftreten.
Denn aus dem \textit{PBSC} kommen zwei Strahlen, welche um $\qty{90}{\degree}$ zueinander polarisiert sind, weshalb diese noch nicht interferieren können. Mit einem Filter um $\qty{45}{\degree}$
tritt das Phänomen dann aber auf. Nun wird zunächst der Strahlengang perfekt angepasst, sodass Interferenzeffekte sichtbar sind und ein möglicher auftretender Streifen im Muster
verschwindet. Nun können die Strahlengänge getrennt werden, indem man den Spiegel \textit{M2} mit der Stage verschiebt, sodass der Laser nicht mehr mittig auf den \textit{PBSC} trifft.
Dadurch entstehen zwei verschobene Vierecke, welche jedoch dennoch im gleichen Punkt enden. Dann wird noch eine Doppelglasplatte eingebaut, welche auf einem Rotationshalter steht. Mit 
dieser kann die Phase der Strahlen beeinflusst werden, sodass maximale und minimale Intensitäten erreicht werden können. Ist nun alles angepasst kann man zunächst einmal den hinteren 
Polarisationsfilter entfehrnen und beide Dioden am Oszilloskop betrachten. Dort sollte die beiden Strahlen Phasenverschoben angezeigt werden. Ist dies der Fall ist der Aufbau korrekt
und messbereit. Für besondere Stabilität der Messung kann eine Haube auf den Aufbau gesetzt werden, welcher Luftschwankungen minimiert.

\section{Bestimmung des Kontrastes eines Sagnac-Interferometers}
\label{sec:kontrast}
Zur experimentellen Bestimmung des Kontrastes soll zunächst nur eine Photodiode verwendet werden. Der Kontrast wurde bereits in \autoref{sec:Kontrast} diskutiert und ist durch Formel \ref{eqn:kontrast}
gegeben. Der Kontrast soll nun in Abhänigkeit des Polarisationswinkels des ersten Polarisators gemessen werden. Dabei soll ein Winkelbereich von $\qty{0}{\degree}-\qty{180}{\degree}$ in 
$\qty{15}{\degree}$-Schritten abgemessen werden. Für jeden Schritt wird die maximale Intensität $I_\mathrm{max}$ und die minimale Intensität $I\_mathrm{min}$ gemssen. Diesen können durch 
die eingebaute Doppelscheibe erzeugt werden und an der Diode abgelesen werden. Aus dieser Messung wird dann der maximale Kontrast bestimmt und der Polarisationsfilter dann auf den 
reziproken Winkel eingestellt.

\section{Bestimmung des Brechungsindex von Glas}
\label{sec:n_glas}
Nun wird die Differenzspannungsmethode verwendet um den Brechungsindex von Glas zu bestimmen. Dazu werden nun beide Photodioden verwendet. Die Differenzspannungsmethode ist eine stabile
und robuste Methode, da mit dieser Störeffekte rausgerechnet werden. Zunächst werden die beiden Dioden an das Oszilloskop angeschlossen und dann die Einstellungen des Interferometry 
Contollers verändert, sodass das Signal sehr deutlich ist. Dies ist nötig, da ansonsten Rauschen für falsche Zählraten sorgt. Die Differenzspannungsmethode verwendet nun dass durch
die beiden Photodioden die Strahlen jewils getrennt voneinander aufgenommen werden. Das heißt, dass die Strahlen auch den exakt selben Störungen ausgesetzt waren. Durch bildung der 
Differenz der beiden Signal, was der Interferometry Contoller macht, kann dann eine Zählung der Intensitätsmaxima bzw. Minima stattfinden. Um nun den Brechungsindex von Glas zu bestimmen 
sollen nun genau die Interferenzmaxima gezählt werden, welche entstehen, während man den Glashalter dreht. Durch die Drehung entsteht ein Phasenversatz wodurch dann die Maxima entstehen.
Es werden nun 10 Messungen über einen Winkelbereich von $\qty{10}{\degree}$ durchgeführt.

\section{Bestimmung des Brechungsindex von Luft}
\label{sec:n_luft}
Zuletzt wird nun der Brechungsindex von Luft bestimmt. Dazu wird zunächst eine Gaszelle in einen der Strahlen eingebaut. Dann wird die Zelle evakuiert bis der Druck auf circa $\qty{0.5}{\milli\bar}$
gefallen ist. Dann wird langsam Luft in die Zelle gelassen und dabei die Anzahl der Intensitätsmaxima gemessen. Es sollten $\qty{50}{\milli\bar}$-Schritte gemacht werden. Der Counter darf 
während der Messung nicht zurückgesetzt werden. Diese Messung wiederholt man drei mal und notiert sich jeweils die Raumtemperatur, da diese für die Analyse wichtig ist.\cite{v64}
