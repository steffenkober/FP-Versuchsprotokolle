\chapter{Diskussion}
\label{cha:Diskussion}
Im ersten Teil des Versuches wurde der Kontrast des Interferometers bestimmt. Der Kontrast hängt von der Justierung des Interferometers ab und sollte möglichst nach an $K = 1$ sein.
Der maximale Kontrast der durchgeführten Justierung liegt bei $K_\mathrm{max} = 0.9608$. Dieser Wert ist sehr nach an einer idealen Justierung, daher kann von einem qualitativen Aufbau
ausgegangen werden von welchen auch qualitative Messwerte erwartet werden können. Dies zeigt sich im nächsten Teil des Versuches. 

Nun wurde der Brechungsindex von Glas bestimmt. Dabei
ergibt sich bei einem Messwert von $\num{1.5228 \pm 0.0124}$ eine relative Abweichung zum Theoriewert von $\Delta n_\mathrm{Glas} = \qty{0.2}{\percent}$. Dies ist eine sehr kleine Abweichung, 
welche allerdings nur gilt, wenn das verwendete Glas normales Fensterglas ist. Leider liegt keine Information darüber vor, weshalb diese Annahme getroffen wurde.

Im letzten Teil des Versuches wurde der Brechungsindex von Luft bestimmt. Hier ergibt sich eine relative Abweichung zum Literaturwert von $\Delta n_\mathrm{Luft} = \qty{1.5795e-05}{\percent}$.
Dieser Abweichung ist ebenfalls sehr gering. Die kleinen Abweichungen können die gute Justierung bestätigen. 

Nichts desto trotz sorgen kleine Abweichung in der äußeren Bedingungen oder kleine Positionsänderungen der Spiegel und Würfel für relativ große Änderungen im Interferenzeffekt. Daher 
bietet sich hier eine mögliche Quelle für Unsicherheiten. Außerdem können, obwohl der Aufbau durch eine Purgebox vor Luftschwankungen geschützt wurde, Luftschwankungen einen Einfluss haben.
Ebenfalls kann ein systematischer Fehler in Form von nicht perfekter Einstellung der DIfferenzspannungsmethode auftreten wodurch Maxima nicht vollständig aufgenommen werden. Dies liegt 
daran, dass auch hintergrundrauschen für mögliche Nulldurchgänge sorgen kann.

Zusammengefasst wurde das Experiment durch alle möglichen Schutzmaßnahmen vor äußeren Einflüssen bewahrt und das Experiment wurde erfolgreich durchgeführt und liefert zufriedenstellende Ergebnisse. 
