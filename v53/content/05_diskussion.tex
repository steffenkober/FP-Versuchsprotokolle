\chapter{Diskussion}
\label{cha:Diskussion}
\section{Untersuchung der Moden im Hohleiter}
Als erstes wurden die Moden des Hohlleiters mit einem Oszilloskop untersucht. Dabei wurden drei unterschiedliche Moden aufgenommen. Dann wurde ein Polynom zweiten Grades an die 
Messwerte aus Tabelle \ref{tab:versuch1} gefittet. Dieser Fit repräsentiert die aufgenommenen Moden. Er ist in Abbildung \ref{fig:moden} dargestellt. Die Frequenzbäder wurden 
ebenfalls aus diesen Messdaten bestimmt. Dazu wurde ein Polynom dritten Grades an die Frequenzmesswerte gefittet. Die Frequenzbänder sind ebenfalls in Abbildung \ref{fig:moden} 
dargestellt. Außerdem wurde aus diesen Messwerten die Abstimmempfindlichkeit berechnet. Diese ergeben sich zu $F_1 = \qty{1.4}{\mega\hertz\per\volt}$, 
$F_2 = \qty{2.05}{\mega\hertz\per\volt}$ und $F_3 = \qty{2.65}{\mega\hertz\per\volt}$. Obgleich die Moden nicht perfekt getroffen wurden, wie in Abbildung \ref{fig:moden} zu sehen ist, 
weisen die steigenden Abstimmempfindlichkeiten einen sinnvollen Verlauf auf. Besonders fällt auf, dass die erste und dritte Mode nicht genau im Maxium gmessen wurde, weshalb es im 
folgendem zu Ungenauigkeiten kommen kann.
\section{Frequenzmessung}
Als nächstes wurde die Frequenz auf zwei Arten bestimmt. Es ergeben sich die Frequenzen $f_1 = \qty{9,103}{\giga\hertz}$ und $f_2 = \qty{8,885 +- 0,01}{\giga\hertz}$. Dabei ist es 
nicht möglich eine Unsicherheit für $f_1$ zu bestimmt. Die Frequenzen liegen nicht sehr nah beieinander, weshalb auf eine systematische Ungenauigkeit geschlossen werden muss. Diese 
kann in diesem Versuch an viele Stellen eingeflossen sein. Zum Beispiel ist die Nadel am SWR-Meter nie wirklich stabil. Außerdem sind sämtliche Einstellungen nicht hoch präzise.
Aus diesen Frequenzen lassen sich die Phasengeschwindigkeiten $v_{\mathrm{ph, 1}} = \qty{4,443(0,005)e8}{\metre\per\second}$ und 
$v_{\mathrm{ph, 2}} = \qty{4,552e8}{\metre\per\second}$ berechnen. Neben den eben diskutierten Abweichungen voneinander fällt auf, dass diese Geschwindigkeiten oberhalb der 
Lichtgeschwindigkeit liegen. Allerdings ist dies bei Phasengeschwindigkeiten, welche bei dispersiven Wellen von der Gruppengeschwindigkeit abweicht, erlaubt.
\section{Messung der Dämpfung}
Danach wurde die Dämpfung durch den Abschwächer mit den Herstellerangaben verglichen werden. Die Regression der Herstellerangaben werden zusammen mit den Messdaten in Abbildung 
\ref{fig:Hersteller} dargestellt. Es ergibt sich eine mittlere Abweichung zu den Herstellerangaben von $4\%$. Diese Abweichung ist genügend klein und zu erwarten aufgrund der 
Einstellmöglichkeit der Schraube.
\section{Messung des SWR}
Zuletzt wurde die SWR auf drei unterschiedliche Arten gemessen. Zunächst wurde über die direkte Methode gemessen. Diese Daten sind in Tabelle \ref{tab:Direkt} dargestellt. Die Steigung
mit zunehmender Schraubentiefe ist zu erwarten. Danach wurde dann die $\qty{3}{\decibel}$-Methode angewendet. Die Messdaten, sowie $S$ sind in Tabelle \ref{tab:3db} dargestellt. Die
verwendete Gleichung gilt lediglich für $S>10$. Da ein Wert $S = \num{12.24}$ bestimmt wurde ist die anwendung der Gleichung legitim. 
Zuletzt wurde $S$ über die Abschwächer Methode bestimmt. Die Daten werden in Tabelle \ref{tab:Abschw} dargestellt. Hier hat sich ein $S = \num{5.62}$ bestimmt. Dieser Wert ist sehr 
klein, weshalb von einem systematischem Fehler ausgegangen werden muss. Ein realistisches Ergebnis sollte $< 10$ sein. Diese erwartung konnte nur mit der $\qty{3}{\decibel}$-Methode
bestätigt werden. Mögliche Fehlerquellen können eine Nichterfassung von Maxima oder Minima sein.
\section{Zusammenfassung}
Insgesamt sind die Messergebnisse nicht ausgezeichnet. Dennoch konnte man bei der Durchführung, das Ziel der Wissenserlangung erfüllen. Das SWR-Meter war aufgrund von starken
Schwankungen schwierig einzustellen. Insgesamt konnte aber das Ziel des Versuches zu einem gewissen Grad erfüllt werden. 

