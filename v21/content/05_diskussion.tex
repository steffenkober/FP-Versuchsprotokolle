\chapter{Diskussion}
\label{cha:Diskussion}
Als erstes wurde das Erdmagnetfeld durch Kompensation bestimmt. Dabei ergab sich ein gemessenes Feld von $B_\mathrm{Erde} = \qty{34.94}{\micro\tesla}$. Der Theoriewert in 
Deutschland lautet $\qty{30}{\micro\tesla}$ \cite{erdmagnetfeld}. Das entspricht einer Abweichung von $\qty{16.47}{\percent}$. Die Kompensation des Erdmagnetfeldes ist hinreichend 
genau. Die Abweichung kann durch ungenaues Ablesen am Oszilloskop entstanden sein. Generell bringt ein Oszilloskop Ungenauigkeiten durch Auflösung und Rauschen mit sich. 
Dennoch sollte die Einstellung der Spulen ausreichend sein um den Versuch durchzuführen.

Danach wurde der Kernspin der Rubidium-Isotope berechnet. Der Kernspin von \ce{^{87}_{37}Rb} wurde zu $I_{87} = \num{1.534(0.007)}$ bestimmt. Der zugehörige Theoriewert des Kernspins 
lautet $I_{87,\mathrm{Theo}} = \frac{3}{2}$. Der Kernspin wurde als mit einer Abweichung von $\qty{2.2(0.5)}{\percent}$ bestimmt. Für \ce{^{85}_{37}Rb} wurde ein Kernspin von 
$I_{85} = \num{2.547(0.008)}$ bestimmt. Hier lautet der THeoriewert $I_{85,\mathrm{Theo}} = \frac{5}{2}$. Dies entspricht einer Abweichung von $\qty{1.89(0.32)}{\percent}$.
Hier kann erkannt werden, dass die Bestimmten zwar nah an der Theorie liegen, allerdings aber nicht im Fehlerintervall liegen. Die kann an zusätzlichen Ungenauigkeiten liegen, wie 
zum Beispiel Fehler durch das Oszilloskop, Erdmagnetfeld und nicht perfekte Justierung des Aufbaus. Daher ist das Unsicherheitsintervall nicht aussagekräftig. Dennoch liegt der 
gemessene Wert sehr nah an dem Theoriewert, weshalb von einer qualitativen Messung sprechen kann. Es ist daher auch zu erwarten, dass ein korrektes Fehlerintervall nicht zu groß wäre. 
Das Isotopenverhältnis wurde zu $\num{0.49}$ bestimmt. Nach Angabe des Betreuers soll das Verhältnis bei ca $0.5$ liegen, aufgrund von Anreicherungen für ein besseres Ergebnis.
Auch das Isotopenverhältnis wurde sehr genau bestimmt. 

Insgesamt wurde die Theorie im Rahmen der Messunsicherheit bestätigt und ein qualitatives Ergebnis erlangt.
