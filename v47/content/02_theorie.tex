\chapter{Theorie}
\label{cha:Theorie}
As mentioned in \autoref{cha:zielsetzung}, we want to understand the nature of the specific heat capacity of Copper. This requires an understanding of the underlying effects, that
influence the heat capacity. But it is more important to first understand what the heat capacity is. The Theory of the heat capacity will be discussed first in a calssical approach
and then in a qunantized approach. %this can be written better 
\section{Heat Capacity}
\label{sec:heat_capacity}
The heat capacity is defined as the amount of energy, which is needed to heat a body by $\qty{1}{\kelvin}$. That means the heat capacity is an amount of heat per temperature. The 
general formula is given by the equation $\ref{eqn:heat_cap}$, in which $Q$ is the needed amount of heat and $\Delta T$ is the temperature change.
\begin{equation}
    C = \frac{Q}{\Delta T}
\end{equation} 