\chapter{Discussion}
\label{cha:discussion}
The calculated specific heat capacity at constant volume plotted against the temperature roughly yields
what is expected from the Debye model. Big uncertainties and measurement outliers would lose importance 
if more data would be taken. This could fix the shape of the plot and could validify the Debye model.\\
The Debye temperature calculated with the measurement data is $\theta_{D,exp} =288.67 \pm 8.62 \, \unit{\kelvin}$ while the theoretically
calculated Debye temperature is $\theta_{D,theo} = 332, 18 \, \unit{\kelvin}$. Therefore those two values do not
overlap and the Debye model cannot be validated. \\
Possible reasons why measurement data is faulty, would be that the heat loss, which was assumed to be rather
small, is actually influencing the data to a significant amount. A better isolation and an automated heat controller
could make the results better.