\chapter{Data Analysis}
\label{cha:Auswertung}
The data was measured with the setup discussed in chapters \ref{cha:Theorie} and \ref{cha:Durchführung}. The duration of the data collection was $T_\mathrm{Dur} = \qty{70.26}{\hour}$. In this 
time $N_{\mathrm{start}} = \num{3427476}$ start signals and $N_{\mathrm{stop}} = \num{1818}$ stop signals were detected by the setup. The search time was set to 
$T_\mathrm{search} = \qty{10}{\micro\second}$.

\section{Estimation of a Constant Background Detection}
\label{sec:background}
A constant background rate in the counting is expected in this experiment. Before conducting the measurement this background rate can be estimated to later evaluate the experimentally 
determined background rate. To estimate the background rate in this experiment the mean amount of muons in the detector has to be calculated. It is given by 
\begin{equation}
    \label{eqn:muon_amount}
    \nu = \frac{N_\mathrm{start}}{T_\mathrm{Dur}} = \qty{13.55}{\per\second} .
\end{equation}
Since this is a counting experiment the counts should be possion distributed and the probability, that $k$ muons hit the detector while it is measering is given by
\begin{equation}
    \label{eqn:prob}
    P_{T_\mathrm{search}\nu}(k) = \frac{(T_\mathrm{search}\nu)^k}{k!}\mathrm{e}^{T_\mathrm{search}\nu} .
\end{equation}

The function \ref{eqn:prob} is proportional to the factorial of $k$, so every order above $k=1$ is negligible. The propability of one additional muon entering the chamber in the 
search time is 
\begin{equation*}
    P_{T_\mathrm{search}\nu}(k=1) = \qty{0.013}{\percent} .
\end{equation*}

From this the estimation of the background rate $U_0$ is given by 
\begin{equation}
    \label{eqn:background}
    U_0 = \frac{N_\mathrm{start}P_{T_\mathrm{search}\nu}(k=1)}{N_\mathrm{channel}} = \qty{1.08}{counts\per channel} .
\end{equation}

In this equation $N_\mathrm{channel} = 430$ is the amount of channels.

\section{Determination of the Resolution Time}
\label{sec:resoltion time}
The resolution time is an important quantity of the setup. It characterizes the minimal temporal distance to distinguish two possible signals.
To determine the resolution time of the setup a delay time $T_\mathrm{Delay}$ is introduced to the two different photomultipliers. This is necessary to phase the photomultipliers,
since signals are only counted when signals of both multipliers arrive in the coincidence at the same time. As a counting experiment the data should comply with the poisson distribution. Hence, the errors are given 
by $\sqrt{N}$. The measered counts dependend on the delay time are shown in table \ref{tab:t_d}. In this table a negative delay time corresponds to the delay of the first 
photomultiplier and a positive delay time to the delay of the second one. 

\begin{table}
    \tiny
    \centering
    \caption{Measured Counts $N$ dependend on the delay time $T_\text{D}$. $T_\mathrm{N}$ is the duration of the measurement. A negative delay time corresponds to one photomultiplier
    and a positive delay time to the second one.}
    \label{tab:t_d}
    \begin{tabular}{c c c}
      \toprule
      {$ T_\text{D} \mathbin{/} \unit{\nano\second}$} & {$N$} & {$T_\mathrm{N} \mathbin{/} \unit{\second}$} \\
      \midrule
    0  & 153 & 20 \\
    1  & 159 & 20 \\
    2  & 146 & 20 \\
    3  & 131 & 20 \\
    4  & 148 & 20 \\
    5  & 136 & 20 \\
    6  & 114 & 20 \\
    7  & 85 & 20 \\
    8  & 65 & 20 \\
    9  & 36 & 20 \\
    10 & 19 & 20 \\
    11 & 22 & 40 \\
    12 & 8  & 60 \\
    13 & 3  & 60 \\
    -1 & 148 & 20 \\
    -2 & 153 & 30 \\
    -3 & 104 & 20 \\
    -4 & 69 & 20 \\
    -5 & 55 & 20 \\
    -6 & 61 & 20 \\
    -7 & 40 & 20 \\
    -8 & 14 & 20 \\
    -9 & 16 & 40 \\
    -10& 11 & 60 \\
    -11& 3 & 60 \\
    \bottomrule
    \end{tabular}
\end{table}

\begin{figure}
    \centering
    \includegraphics[width = .8\textwidth]{plot1.pdf}
    \caption{The dependece of the count rate to the delaytime. Negative delay corresponds to the delay of one cable and positive to the other cable. The dashed line shows the 
    full width half at half maximum and the continuous blue line shows the plateau.}
    \label{fig:t_d}
\end{figure}

The data is also shown in figure \ref{fig:t_d}. In this figure the course of the data shows a plateau from $\qty{-1}{\nano\second}$ to $\qty{5}{\nano\second}$ at around $\qty{7.2}{\second^{-1}}$.
This plateau arises from the set pulse width of $\qty{10}{\nano\second}$. This leads to no change in the signal counts, because the delay is smaller than the pulse width. But due 
to the not-constant shape of the pulses the plateau is not $\qty{10}{\nano\second}$ wide.
The resolution time corresponds to the full width at half maximum (FWHM) of the data. The FWHM is the width of the distribution at half of the maxiumum value.
The FWHM can be determined graphically by plotting a horizontal line at half the value of the plateau.
The crossings of this line with the data lead to a linewidth of the horizontal line by cutting off values outside of the data curve. This linewidth is interpreted as the FWHM.
As shown in figure \ref{fig:t_d} the resolution time for this setup is $\qty{12}{\nano\second}$. 

\section{Calibration of the Multichannel Analyser}
\label{sec:MCA}

As discussed earlier the TAC converts the temporal distance of the signals into an amplitude. The multichannel analyser (MCA) converts this amplitude into counts amoung different 
channels. The number of the categorized channel has a correlation to the measured lifetime of the muons. This correlation has to be determined before the measurement. The measured 
data of the time dependece of the MCA is given in table \ref{tab:MCA}. For better visualization the data is plotted in figure \ref{fig:MCA}. 

\begin{table}
    \tiny
    \centering
    \caption{Channel number correspondend to the temporal pulse distance.}
    \label{tab:MCA}
    \begin{tabular}{S[table-format = 3.0] S[table-format = 1.1]}
      \toprule
      {Kanal} & {$t \mathbin{/} \unit{\micro\second}$} \\
      \midrule
       16 & 0,5 \\
       39 & 1,0 \\
       62 & 1,5 \\
       85 & 2,0 \\
      108 & 2,5 \\
      131 & 3,0 \\
      154 & 3,5 \\
      177 & 4,0 \\
      200 & 4,5 \\
      223 & 5,0 \\
      246 & 5,5 \\
      269 & 6,0 \\
      292 & 6,5 \\
      315 & 7,0 \\
      338 & 7,5 \\
      361 & 8,0 \\
      384 & 8,5 \\
      407 & 9,0 \\
      430 & 9,5 \\
      \bottomrule
    \end{tabular}
\end{table}

\begin{figure}
    \centering
    \includegraphics[width = .7\textwidth]{plot2.pdf}
    \caption{Linear regression\cite{scipy} of the data to obtain a functional correlation between the channel number and the lifetime.}
    \label{fig:MCA}
\end{figure}

A linear regression of the data is used to calculate parameters, which determine the functional correlation \ref{eqn:MCA} between the channel number $K$ and the time $t$.

\begin{equation}
    \label{eqn:MCA}
    t[\unit{\micro\second}] = 0.021739K +  0.15217
\end{equation}

\section{Lifetime of Cosmic Muons}
\label{sec:lifetime}

The program used for the counting yields a data file in which the counts of each channel are saved. With the conversion function \ref{eqn:MCA} signal durations are calculated. The data 
is shown in figure \ref{fig:lifetime}.

\begin{figure}
    \centering
    \includegraphics[width = .8\textwidth]{fit.pdf}
    \caption{Data for the determination of the lifetime of cosmic muons. $N$ are the counts of muons and $t$ is time correspondend to the channel.}
    \label{fig:lifetime}
\end{figure}

The counts of the muon decays should follow an exponential law given by equation \ref{eqn:exp}. 

\begin{equation}
    \label{eqn:exp}
    N(t) = N_0 \mathrm{e}^{-t/\tau} + U_0
\end{equation}

In this equation $N_0$ is the basis amount of muons, $\tau$ is the lifetime and $U_0$ is the constant background rate. $U_0$ was estimated in section \ref{sec:background}.
To extract the lifetime $\tau$ the exponential function given by equation \ref{eqn:exp} is fitted to the data in figure \ref{fig:lifetime}.
The fit yields the properties of the dataset. The values are:
\begin{align*}
    N_0 &= \num{18.9 +- 0.7}, & \tau &= \qty{1.62 +- 0.11}{\micro\second}, & U_0 &= \num{1.4 +- 0.2}
\end{align*}

The regression is also plotted in figure \ref{fig:lifetime}.