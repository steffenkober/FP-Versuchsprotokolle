\chapter{Data Analysis}
\label{cha:Auswertung}
The data was measuered with the setup discussed in the chapters \ref{cha:Theorie} and \ref{cha:Durchführung}. The duration of the data collection was $\qty{70.26}{\hour}$. In this 
time $N_{\mathrm{start}} = \num{3427476}$ start signals and $N_{\mathrm{stop}} = \num{1818}$ stop signals were detected by the setup. The search time was set to 
$T_\mathrm{search} = \qty{10}{\micro\second}$.

\section{Determination of the Resolution Time}
\label{sec:resoltion time}
To determine the resolution time of the setup a delay time $T_\mathrm{Delay}$ is introduced to the two different photomultipliers. This is necessary to phase the photomultipliers,
since signals are only counted when both multipliers give a signal. As a counting experiment the data should comply with the possion distribution. Hence, the errors are given 
by $\sqrt{N}$. The measerd counts dependet on the delay time are shown in table \ref{tab:t_d}. In this table a negative delay time corresponds to one photomultiplier and a positive 
delay time to the second one. 

\begin{table}
    \tiny
    \centering
    \caption{Zählraten $N$ der Koinzidenz gegen verschiedene Verzögerungszeiten $T_\text{VZ}$ der Leitungen. Eine negative Verzögerungszeit beschreibt eine größere Verzögerung der Leitung zum linken PMT.
    Es sind zwei Messreihen zu verschiedenen Zählzeiten aufgetragen.}
    \label{tab:Mess1}
    \begin{tabular}{c c c}
      \toprule
      {$ T_\text{D} \mathbin{/} \unit{\nano\second}$} & {$N$} & {$T_\mathrm{N} \mathbin{/} \unit{\second}$} \\
      \midrule
    0  & 153 & 20 \\
    1  & 159 & 20 \\
    2  & 146 & 20 \\
    3  & 131 & 20 \\
    4  & 148 & 20 \\
    5  & 136 & 20 \\
    6  & 114 & 20 \\
    7  & 85 & 20 \\
    8  & 65 & 20 \\
    9  & 36 & 20 \\
    10 & 19 & 20 \\
    11 & 22 & 40 \\
    12 & 8  & 60 \\
    13 & 3  & 60 \\
    -1 & 148 & 20 \\
    -2 & 153 & 30 \\
    -3 & 104 & 20 \\
    -4 & 69 & 20 \\
    -5 & 55 & 20 \\
    -6 & 61 & 20 \\
    -7 & 40 & 20 \\
    -8 & 14 & 20 \\
    -9 & 16 & 40 \\
    -10& 11 & 60 \\
    -11& 3 & 60 \\
    \bottomrule
    \end{tabular}
\end{table}
