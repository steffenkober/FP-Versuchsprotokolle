\chapter{Experimental aparatus}
\label{cha:Durchführung}

By using a oscilloscope, the functionality of the photomultipliers is checked and the pulse duration is set to be approximately $\Delta t = \qty{10}{\micro\second}$. The threshold od the discriminators
is adjusted, so that approximately 30 pulses per second are measured at both outputs. For this, a counter is used.\\
To adjust the variable delays, different delays are chosen and the countrate is measured. The distribution has a broad plateau, from which a value
in the middle is chosen. The half-width of this distribution is to be determined. The event rate should be at $\qty{20}{\second}^{-1}$ by now.\\
The remaining electronics are connected as shown in \autoref{fig:Aufbau}. A search time of $T_s = \qty{10}{\micro\second}$ is set at the monoflop
and the measuring range of the TAC is adjusted, so that the measured time steps are small enough, but not too detailed.\\
A double pulse generator operating at $\qty{1}{\kilo\hertz}$ is connected to the MCA, to check which time intervals correspond to which channels of the
MCA. \\
After the calibration the measurement is started. The overall measuring time is approximately three days.