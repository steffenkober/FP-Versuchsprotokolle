\chapter{Discussion}
\label{cha:Diskussion}
The in this experiment calculated lifetime of muons is $\tau = \qty{1.62 +- 0.11}{\micro\second}$. The literature value is given by 
$\tau = \qty{2.1969811 +- 0.0000022}{\micro\second}$ \cite{PDG}. This experiment results in an deviation of $\qty{26 +- 5}{\percent}$. This deviation is large and cannot be explained 
with statistical uncertainty. Small uncertainties can rise from muonic atoms, which have a shorter lifetime, or the unknown uncertainty of the calibration of the MCA. But the 
deviation suggests an systematic error in the setup. The setup has many possible origins of systematic errors. The first problem could be the descriminator. If not adjusted equally 
and correctly the amount of data can be reduced drasticly and also a bias can be introduced, which could favor signals from longer or shorter lifetimes. Another problem could be an 
low event rate. It was given, that the event rate should lie around $\qty{20}{\per\second}$. But this event rate was not reached, so that the experiment was executed with an event rate 
of $\propto \qty{15}{\per\second}$. This again reduces the data. Additionally to the already low amount of data the mesurement yielded one channel with extraordinary amount of 
counts. This channel is not included in the analysis due to an obvious error in the counting. Without these data points it is significant, that the low channels have very low or even 
0 counts. This again indicates a systematic error. 

The same uncertainty can be seen in the difference of the predicted background $(U_0 = \qty{1.08}{counts\per channel})$ and the background of the regression $(U_0 = \num{1.4 +- 0.2})$.
All this indicates, that the regression is incorrect and the results are not trustworthy. 

Nevertheless, besides an imperfect data collection, the goal of the experiment was also to achieve knowledge about the measurement concept and the NIM-standard. This goal is still 
achieved and therefore the experiment is partially  successful.
