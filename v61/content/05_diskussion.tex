\chapter{Diskussion}
\label{cha:Diskussion}
Zu Beginn wurde die Stabilitätsbedingung des Laser untersucht. Dabei konnte bei keinem Aufbau die theoretisch maximal mögliche Länge überprüft werden, da in einem Fall kein Signal mehr
gefunden werden konnte und in dem anderem Fall war die Länge nicht realisierbar durch den begrenzten Platz. Dennoch haben die Messungen sinkende Stabilität mit größeren 
Resonatorlängen aufgezeigt. Daher kann angenommen werden, dass ab der maximal möglichen Länge, innerhalb der experimentellen Ungenauigkeit, kein Signal mehr zu finden ist. 
Ebenfalls wurde ein stabiler Laser Betrieb nachgewiesen. 

Als nächstes wurden die Transversalmoden des Lasers untersucht. Dabei wurden die Messdaten an die theoretische Vorhersage gefittet. Wie den Abbildungen \ref{fig:TEM00}, \ref{fig:TEM01}
zu entnehmen ist, bestätigen die Messwerte das theoretische Modell der Intensitätsverteilung.

Daraufhin wurde die Polarisation des Laserstrahls untersucht. Auch hier bestätigt die Messung die theoretische Vorhersage des Modells. In dieser Messung sind die Abweichungen gering, 
weshalb die Messung sehr qualitativ ist.

Weiter wurde der Multimoden Betrieb des Lasers untersucht. Es stellt sich heraus, dass die Schwebungsfrequenzen des Laser sehr viel kleiner sind als die Dopplerbreite des Lasers. 
Daher wurde der Multimoden Betrieb validiert. 

Zuletzt wurde die Wellenlänge des Lasers bestimmt. Es ergibt sich eine experimentell bestimmte Wellenlänge von $\lambda = \qty{635.17 +- 7.40}{\nano\metre}$. Die theoretische 
Vorhersage der Wellenlänge des $3\text{s}_2 \rightarrow 2\text{p}_4$ Übergangs, in der Paschen Notation, lautet $\lambda = \qty{632.8}{\nano\metre}$ \cite{Wikipedia_HeNe}.
Dies entspricht einer relativen Abweichung von $\Delta \lambda = \qty{0.22}{\percent}$. Die theoretische Wellenlänge liegt im Rahmen der Unsicherheit und somit ist die Messung 
der Wellenlänge gelungen. Gründe für abweichende Werte liegen in der Ungenauigkeit der Bestimmung der Abstände, welche händisch mit einem Maßband abgemessen wurden. 

Insgesamt sind die Ergebnisse des Experiments zufriedenstellend und das Experiment kann also erfolgreich gewertet werden.