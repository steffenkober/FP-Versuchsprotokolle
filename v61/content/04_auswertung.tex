\chapter{Auswertung}
\label{cha:Auswertung}

In diesem Kapitel werden die Messdaten ausgewertet. Die Stabilitätsbedingung wird für zwei verschiedene Resonatoren untersucht. Danach wird sich die analyse auf einen Resonator 
fokussieren, welcher aus zwei konkaven Spiegeln besteht.

\section{Stabilitätsbedingung}
\label{sec:stabil}
Zunächst wird die Stabilitätsbedingung für den Laserbetrieb untersucht. Dabei wurde als erstes der Resonator bestehend aus einem planaren Spiegel ($r_1 = \infty$) und einem
konkaven Spiegel ($r_2 = \qty{1.4}{\metre}$) verwendet. 
Aus Formel \ref{eqn:stabil} kann gefolgert werden, dass dieser Resonator eine maximale Länge $L_\text{max} = r_2$ haben darf. Dies bedeutet, dass sobald eine Länge von $L_\text{max}$
überschritten wird ist dier Laser nicht mehr stabil und die Intensität sollte verschwinden. In Tabelle \ref{tab:stabil_plan} sind die aufgenommenen Messdaten dargestellt. 

\begin{table}
    \centering
    
    \begin{tabular}{c c}
        \toprule
        {$ L \mathbin{/} \unit{\centi\metre}$} & {$I \mathbin{/} \unit{\milli\watt}$}\\
        \midrule
        49.8 & 3.60 \\ 
        60.3 & 2.29 \\
        70.2 & 1.61 \\
        80.2 & 1.21 \\
        90.1 & 0.20 \\
        \bottomrule
    \end{tabular}
    \caption{Laserintensitäten in Abhängigkeit von der Resonatorlänge für einen plan/konkaven Resonator des Radius $r_2 = \qty{1.4}{\metre}$.}
    \label{tab:stabil_plan}
\end{table}

Die Tabelle \ref{tab:stabil_plan} stellt lediglich Messdaten bis zu einer Resonatorlänge von $L = \qty{90.1}{\centi\metre}$ dar. In diesem Versuch ist es nicht gelungen ein stabiles 
Signal oberhalb dieser Länge zu erzeugen. Dennoch zeigt sich für große Resonatorlänge auf, dass die Intensität sinkt. Daher ist die Annahme valide, dass sie ab Verletzung der 
Stabilitätsbedingung vollkommen verschwindet.

Weite wurde die Stabilitätsbedingung für einen konkav/konkaven Resonator, der Radien $r_1 = r_2 = \qty{1.4}{\metre}$, untersucht. Aus Formel \ref{eqn:stabil} ergibt sich für diesen 
Resonator eine maximale Länge $L_\text{max} = \qty{2.4}{\metre}$. Die Messdaten für diesen Resonator sind in Tabelle \ref{tab:stabil_konkav} dargestellt. 

\begin{table}
    \centering

    \begin{tabular}{c c}
        \toprule
        {$ L \mathbin{/} \unit{\centi\metre}$} & {$I \mathbin{/} \unit{\milli\watt}$}\\
        \midrule
        49.5  & 5.00 \\
        58.3  & 4.79 \\
        65.8  & 4.46 \\
        77.1  & 4.63 \\
        83.9  & 4.60 \\
        91.6  & 4.20 \\
        105.8 & 3.35 \\
        119.9 & 2.65 \\
        172.7 & 1.07 \\
        \bottomrule
    \end{tabular}
    \caption{Laserintensitäten in Abhängigkeit von der Resonatorlänge eines konkav/konkaven Resonators der Radien $r_1 = r_2 = \qty{1.4}{\metre}$.}
    \label{tab:stabil_konkav}
\end{table}

Auch die Messdaten in Tabelle \ref{tab:stabil_konkav} reichen nicht über die maximale Länge hinaus, da die Länge der Schiene, auf welcher der Aufbau aufgebaut war, nicht ausreichte.
Dennoch zeigen die Daten auch bei der letzten Länge noch ein signifikantes Signal. Ebenfalls sinken die Intensitäten kontinuierlich was darauf hindeutet, dass die Stabilität abnimmt.
Dennoch ist für die Resonatorlängen aus Tabelle \ref{tab:stabil_konkav} nachgewiesen worden, dass ein stabiler Betrieb des Laser gewährleistet ist. Im folgendem wird ausschließlich die
konkav/konave Spiegelkonfiguration verwendet.

\section{TEM Moden}
\label{sec:TEM}
Als nächstes werden die Transversalmoden des HeNe-Lasers untersucht. Im Rahmen der zur Verfügung stehenden Mittel ist lediglich die Observierung der $\text{TEM}_{00}$ und der 
$\text{TEM}_{01}$ möglich. Das Vorgehen hierzu wurde bereits im Abschnitt \ref{sec:TEM_durch} erklärt. 

\subsection{$\text{TEM}_{00}$ Mode}
\label{subsec:TEM00}
Die allgemeine Intensitätsverteilung der $\text{TEM}_{nm}$ Moden wurde bereits im Abschnitt \ref{sec:resonator} hergeleitet. Für die Grundmode ergibt sich aus Gleichung \ref{eqn:intensity} eine 
Intensitätsverteilung nach Gleichung \ref{eqn:I_00}.

\begin{equation}
    \label{eqn:I_00}
    I_{\text{TEM}_{00}} = I_0e^{\frac{-(x-x_0)^2}{2w^2}} 
\end{equation}

Die Messdaten werden mittels Scipy \cite{scipy} an die Gleichung \ref{eqn:I_00} gefittet. Die daraus resultierenden Parameter lauten
\begin{align*}
    I_0 &= \qty{9.56 +- 0.078}{\micro\ampere}\\
    x_0 &= \qty{3.78 +- 0.0038}{\milli\metre}\\
    w   &=  \qty{0.82 +- 0.0078}{\milli\metre}.\\
\end{align*}

Der Fit mit den eben bestimmten Parametern, sowie die Messdaten werden in Abbildung \ref{fig:TEM00} dargestellt.

\begin{figure}
    \centering
    \includegraphics[width = .8\textwidth]{build/TEM_00.pdf}
    \caption{In dieser Abbildung werden die Messwerte sowie die gefittete Intensitätsverteilung in x-Richtung der $\text{TEM}_{00}$ dargestellt. $r$ beschreibt dabei die Drahtposition.}
    \label{fig:TEM00}
\end{figure}

\subsection{$\text{TEM}_{01}$ Mode}
\label{subsec:TEM01}
Nun wird die Intensitätsverteilung der $\text{TEM}_{01}$ Mode bestimmt. Aus Gleichung \ref{eqn:intensity} ergibt sich die erwartete Intensitätsverteilung nach Gleichung \ref{eqn:I_01}.
\begin{figure}
    \centering
    \includegraphics[width = .8\textwidth]{build/TEM_01.pdf}
    \caption{In dieser Abbildung werden die Messwerte sowie die gefittete Intensitätsverteilung in x-Richtung der $\text{TEM}_{01}$ dargestellt. $r$ beschreibt dabei die Drahtposition.}
    \label{fig:TEM01}
\end{figure}
\begin{equation}
    \label{eqn:I_01}
    I_{\text{TEM}_{01}} = I_0\frac{8(x-x_0)^2}{w^2}e^{\frac{-(x-x_1)^2}{2w^2}} 
\end{equation}

An dieses Modell werden die Messdaten mittels Scipy \cite{scipy} gefittet und Modellparameter bestimmt. Die Parameter für die aufgenommenen Daten lauten 

\begin{align*}
    I_0 &= \qty{0.84 +- 0.028}{\micro\ampere}\\
    x_0 &= \qty{5.17 +- 0.008}{\milli\metre}\\
    x_1 &= \qty{5.18 +- 0.009}{\milli\metre}\\
    w   &= \qty{0.56 +- 0.011}{\milli\metre}\\
\end{align*}

Die Messdaten, sowie der Fit des Modells \ref{eqn:I_01} sind in Abbildung \ref{fig:TEM01} dargestellt.



\section{Polarisation}
\label{sec:Polarisation}
Weiter wird die Polarisation des emmitierten Lichtes untersucht. Dazu wird die Intensität in Abhänigkeit der Winkeleinstellung $\Phi$ eines Polarisationsfilters gemessen. 
Für linear polarisiertes Licht ist Winkelabhängigkeit der Intensität $I$ nach Gleichung \ref{eqn:intensity} zu erwarten. Dabei ist $I_0$ die Amplitude des Signals und $\Phi_0$ 
eine beliebige Phase.

\begin{equation}
    \label{eqn:intensity}
    I(\Phi) = I_0 \sin^2(\Phi - \Phi_0)
\end{equation}

Die Messdaten werden mittels Scipy \cite{scipy} an das Modell \ref{eqn:intensity} gefittet und somit Parameter für das Modell bestimmt. Die Parameter für die aufgenommenen Daten lauten 

\begin{align*}
    I_0 &= \qty{2.98 +- 0.005}{\milli\watt}\\
    \Phi_0 &= \qty{-0.37 +- 0.0015}{rad}\\
\end{align*}

\begin{figure}
    \centering
    \includegraphics[width = .8\textwidth]{build/Polarisation.pdf}
    \caption{In dieser Abbildung wird die gemessene Intensitätsverteilung, sowie das gefittete Modell in Abhängigkeit des Polarisationswinkles $\Phi$ dargestellt.}
    \label{fig:Polarisation}
\end{figure}

Die Messdaten sowie der Fit werden in Abbildung \ref{fig:Polarisation} dargestellt.


\section{Multimoden Betrieb}
\label{sec:Multimoden}
Nun soll der Multimoden Betrieb des Laser untersucht werden. Dafür wird das Signal des Laser an einem Oszilloskop dargestellt. Für verschiedene Resonatorlängen wird eine Vielzahl an 
Frequenzpeaks im Oszilloskop observiert. Die mittlere Schwebungsfrequenz kann dabei aus dem mittleren Abstand zweier benachbarter Peaks zueinander berechnet werden. 
Das Signal des HeNe-Lasers sollte eine gesamte Frequenzbreite von $\approx f_\text{HeNe} = \qty{1.5}{\giga\hertz}$ haben. Dies entspricht der Dopplerbreite, welcher der dominierende Effekt
für das ausbreiten des Signals ist. Die Anzahl der Moden $N$ im Laser ist also korriliert zur Schwebungsfrequenz auf einen festen Wert beschränkt, jenachdem wie viele Schwebungsfrequenzen 
in die Dopplerbreite passen. In Tabelle \ref{tab:multimoden} sind für verschiedene Resonatorlängen die gemessenen Frequenzen, sowie die berechnete Schwebungsfrequenz und die Anzahl der 
Moden dargestellt.

\begin{table}
    \centering
    \begin{tabular}{c c c c}
        \toprule
        {$ L \mathbin{/} \unit{\centi\metre}$} & {$f \mathbin{/} \unit{\mega\hertz}$} & {$\Delta f \mathbin{/} \unit{\mega\hertz}$} & {$N$}\\
        \midrule
         63.0  & 240,476,716,953,1189 & 237.25 & 8.80 \\
         77.4  & 195,386,581,773,968,1163,1354 & 210.8 & 8.65 \\
         87.0  & 173,345,518,690,863,1035,1208,1380 & 195.0 & 8.65 \\
        105.8  & 143,285,428,570,713,851,994,1136,1279 & 178.04 & 8.87 \\
        119.9  & 128,251,375,503,626,754,878,1001,1125,1253,1376 & 162.83 & 9.21 \\
        \bottomrule
    \end{tabular}
    \caption{Diese Tabell stellt die Frequenzpeaks im Oszilloskop in Abhängigkeit der Resonatorlänge dar.}
    \label{tab:multimoden}
\end{table}

Aus Tabelle \ref{tab:multimoden} kann entnommen werden, dass Schwebungsfrequenz mit steigender Resonatorlänge abnimmt. Dieses Verhalten ist zu erwarten, da die Laufzeit der Photonen 
steigt und somit die Frequenz sinkt. Ebenfalls kann entnommen werden, dass die Schwebungsfrequenz für alle Resonatorlängen sehr viel kleiner ist als die Dopplerbreite weshalb der 
Multimoden Betrieb des Laser bestätigt und validiert wird. 

\section{Wellenlänge}
\label{sec:Wellenlänge}
Zuletzt wird die Wellenlänge des HeNe-Laser untersucht. Dazu wird ein Gitter verwendet. Die Beugung am Gitter ist durch die Bragg-Bedingung beschrieben. Durch umstellen der 
Bragg-Bedingung und Berechnung den Winkel zwischen der optischen Achse und der Verbindungsliene von Beugungsmaximum und dem Gitter kann die Wellenlänge nach Gleichung \ref{eqn:wellenlaenge}
berechnet werden.

\begin{equation}
    \label{eqn:wellenlaenge}
    \lambda = \frac{\sin\left(\tan^{-1}\left(\frac{s_n}{d}\right)\right)}{gn}
\end{equation}

In dieser Gleichung ist $s_n$ der minimale Abstand des $n$-ten Beugungsmaximum zur optischen Achse. $d$ ist der Abstand vom Gitter zum Schirm, $g$ die Gitterkonstante des verwendeten 
Gitters und $n$ die Ordnung des Beugungsmaximums. Die Messwerte werden in Tabelle \ref{tab:wellenlaenge} dargestellt.  

\begin{table}
    \centering
    \begin{tabular}{c c c c}
        \toprule
        {$n$} & {$s \mathbin{/} \unit{\centi\metre}$} & {$g \mathbin{/} \unit{\milli\metre}^{-1}$} & {$\lambda \mathbin{/} \unit{\nano\metre}$}\\
        \midrule
         1  &  8.9 &  600 & 642.56 \\
         1  & 24.4 & 1200 & 627.78 \\

        \bottomrule
    \end{tabular}
    \caption{Diese Tabelle stellt die Messdaten zur Bestimmung der Wellenlänge dar. Dabei ist $n$ die Beugungsordnung, $s$ der minimale Abstand des $n$-ten Beugungsmaximum zur optischen Achse,
    $g$ die Gitterkonstante und $\lambda$ die Wellenlänge.}
    \label{tab:wellenlaenge}
\end{table}

Der Mittelwert der errechneten Wellenlängen ergibt sich zu 
\begin{equation*}
    \lambda = \qty{635.17 +- 7.40}{\nano\metre}
\end{equation*} 
