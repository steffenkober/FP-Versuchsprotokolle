\chapter{Diskussion}
\label{cha:Diskussion}

Der bestimmte Wert für den Schwellenstrom $I_{\mathrm{thr}} = \qty{35.1}{\milli\ampere}$ stellt einen plausiblen Wert dar. Ein Vergleich
zu einem Theoriewert kann nicht gezogen werden.\\
Die Fluoreszenz des Rubidiums ist deutlich sichtbar, obgleich die Qualität des aufgenommenen Bildes den beobachteten Effekt nicht in ganzer
Intensität widergibt. Es wurde ein Flackern der Fluoreszenz mit steigenden Strom beobachtet. Dies ist ein zu erwartender Effekt.\\
Das aufgenommene Absorptionsspektrum des Rubidium-Gasgemisch lassen sich sehr gut der theoretischen Vorhersage zuordnen. Das Verhältnis der
Isotope lässt sich ebenfalls abschätzen und liegt im Einklang mit dem in der Natur vorliegenden Verhältnis.