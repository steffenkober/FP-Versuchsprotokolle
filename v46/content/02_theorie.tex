\chapter{Theorie}
\label{cha:Theorie}
Eine generelle Einleitung in die benötigten theoretischen Grundlagen wird im folgenden Kapitel präsentiert. Dazu gehört
die Bandstruktur der Halbleiter, das Konzept der effektiven Masse und die Faraday-Rotation. 
\section{Die Bandstruktur eines Halbleiters}
Das Modell der Bandstruktur beschreibt die Überlagerung diskreter Energieniveaus der Elektronen in einem Kristall, hervorgerufen durch
ein gitterperiodisches Potential. Diese Überlagerung verursacht die Entstehung von Energiebändern und Bandlücken. Das Valenzband ist jenes, in dem Elektronen gebunden existieren,
während jene im Leitungsband als delokalisiert angenommen werden können. Ob und unter welchen Bedingungen Elektronen im Leitungsband existieren können,
bestimmen die Eigenschaften eines Materials.\\
Kristalle mit einer großen Bandlücke zwischen dem Valenz- und Leitungsband sind Isolatoren. Elektronen aus dem Valenzband können unter keinen Bedingungen in 
das Leitungsband angeregt werden. Für Metalle überlappen sich beide Bänder, wodurch das Material dauerhaft leitend wird. Falls eine Bandlücke zwischen 
Leitungs- und Valenzband existiert, welche klein genug ist ($E_{\text{gap}}=\qtyrange{0}{6}{\electronvolt}$) um durch eine Anregung überwunden zu werden, wird das entsprechende Material zu einem Halbleiter. 
Hier kann bereits eine Anregung über thermische Strahlung reichen, also einer Umegbungstemperatur von über $\qty{0}{\kelvin}$.
Hierbei unterscheidet man zwischen direkten und indirekten Halbleitern, bei denen entweder eine energetische Anregung oder eine zusätliche Anregung über einen 
Impuls nötig ist. Eine schematische Darstellung der Bandstrukturen von Isolatoren, Metallen und Halbleitern ist in \autoref{fig:Bandstruktur} dargestellt.
\begin{figure}
    \centering
    \includegraphics[width = \textwidth]{content/V46_pictures/Bandstruktur.png}
    \caption{Schematische Darstellung der Bandstruktur von Isolatoren (links), Metallen (mitte) und Halbleitern (rechts). $E_F$ bezeichnet hier die Fermienergie. \cite{grossmarx}}
    \label{fig:Bandstruktur}
\end{figure}
\\Die Leitfähigkeit eines Halbleiters kann durch eine Dotierung, der Einbringung eines Fremdatoms manipuliert werden. Bei den Atomen wird zwischen Donatoren (bei einer n-Dotierung) und
Akzeptoren (bei einer p-Dortierung) unterschieden.\\
Ein Donator gehört einer Ordnungsgruppe als die Kristallatome, wodurch das überschüssige Elektron des Fremdatoms nur leichte Kräfte 
erfährt und als delokalisiert betrachtet werden kann. Dadurch reichen bereits kleine Energien aus, um das Material leitend zu machen.
Akzeptoren gehören einer niedrigeren Ordnungsgruppe an als die Atome des Kristalls, wodurch ein Überschuss an positiver Ladung folgt.\\
Der in diesem Experiment untersuchte Kristall ist n-dotiertes Galliumarsenid (GaAs), welches ein direkter Halbleiter ist. Galliumarsenid hat ene Bandlücke von $E_{\text{gap, GaAs}} = \qty{1,42}{\electronvolt}$,
wodurch es bei Raumtemperatur nur leicht leitend ist. Dieser Effekt wird durch die Dotierung verstärkt.

\section{Effektive Masse}
Dadurch, dass auf die Elektronen in einem Kristall ein gitterperiodisches Potential wirkt, müssen diese Effekte auf die Bewegung des Elektrons bei der Beschreibung
dessen berücksichtigt werden. Allerdings da das Potential periodisch ist, lässt sich der Effekt mithilfe dem Konzept der effektiven Masse kompensieren, wodurch es möglich ist
das Elektron in harmonischer Näherung frei zu beschreiben. Hierfür wird die Energie des Leitungsbandes zunächst mithilfe einer Taylorentwicklung zweiter Ordnung abgeschätzt:
\begin{equation*}
    \mathcal{T}_{(E(\mathbf{k}))} = E(0) + \frac{1}{2}\sum_{\symup{i}=1}^{3}\left(\frac{\partial E^2}{\partial k_{\symup{i}^2}}|_{k=0}\right) k_{\symup{i}}^2 + \mathcal{O}(E(\mathbf{k})^3)
\end{equation*}
Mithilfe von $E=frac{\hbar k^2}{2m}$ lässt sich die Gleichung umformulieren zur effektive Masse über
\begin{equation*}
    m_{\symup{i}}^{*}=\frac{\hbar^2}{\left(\frac{\partial E^2}{\partial k_{\symup{i}^2}}|_{k=0}\right)}.
\end{equation*}
Die effektive Masse ist durch das dreidimensionale Potential eines Kristalls eine tensorielle Größe. Bei Kristallen hoher Raumsymmetrie, ist der Tensorcharakter der effektiven Masse
allerdings vernachlässigbar und lässt sich als Skalar beschreiben. Dies ist bei Galliumarsenid der Fall.

\section{Faraday-Rotation}

Eine linear polatisierte Welle besteht aus der Superposition zwei entgegengesetzt zirkular polarisierter Wellen. Durch unterschiedliche Phasengeschwindigkeiten der zirkular polarisierten Wellen
kann es beim eintreten in ein optisches aktiven Medium zu einer Drehung der Polarisationsebene des linear polarisierten Lichts kommen. Dieser Effekt nennt sich zirkulare Doppelbrechung.\\
In optisch inaktiven Medien, wie Galliumarsenid, tritt dieser Effekt unter normalen Umständen nicht auf. Er kann allerdings durch Ausrichtung eines starken Magnetfeldes parallel zu Ausbreitungsrichtung des 
linear polatisierten Lichts erzwungen werden. Die zirkulare Doppelbrechung in optisch inaktiven Medien durch Einbringung eines Magnetfeldes nennt sich Faraday-Rotation und ist schematisch in \autoref{fig:Faraday} gezeigt.
\begin{figure}
    \centering
    \includegraphics[width = 0.5\textwidth]{content/V46_pictures/Faraday.png}
    \caption{Darstellung der Drehung der Polarisationsebenen einer linear polarisierten Wellen, hervorgerufen durch die Faraday-Rotation \cite{wiki_Fara}}
    \label{fig:Faraday}
\end{figure}
\\Der Drehwinkel, um den die Polarisationsebene der Welle beim Eintritt in das Medium rotiert wird, ist durch
\begin{equation}
    \label{eq:theta}
    \theta = \frac{L\omega}{2}\left(\frac{1}{v_{ph, r}} - \frac{1}{v_{ph, l}}\right) = \frac{L\omega}{2c}\left(n_r - n_l\right)
\end{equation}
gegeben. Der Unterschied der Brechungsindize der rechtszirkular polarisierten Welle $n_r$ und linkszirkular polarisierten Welle $n_l$ bestimmt also maßgeblich die Rotation. Ein weiterer
Faktor ist die durchlaufene Strecke im Medium $L$ und die Kreisfrequenz $\omega$. Die Faraday-Rotation ensteht durch magnetisch induzierte elektrische Dipolmomente innerhalb des Kristalle, wodurch eine makroskopische
Polarisation $\vec{P}$ des Stoffs entsteht.\\
Die Suszeptibilität, welche in direkten Zusammenhang mit der Phasengeschwindigkeit der zirkularen Wellen steht, ist von dem angelegten Magnetfeld abhängig über
\begin{equation}
    \chi_\text{xy} = \frac{Ne_0^3\omega B}{\varepsilon_0\left(\left(-m\omega^2+K\right)^2-\left(e_0\omega B\right)^2\right)},
    \label{eq:Suszeptibilitaet_Faraday}
\end{equation}
wobei $\chi_\text{xy}$ die Einträge des Suszeptibilitätstensor sind.\\
Mithilfe von 
\begin{align}
    n_r &= \sqrt{1+\chi_\text{xx}+\chi_\text{xy}} \\
     \text{und } n_l &= \sqrt{1+\chi_\text{xx}-\chi_\text{xy}}
\end{align}
sowie \autoref{eq:Suszeptibilitaet_Faraday} kann \autoref{eq:theta} angenähert werden mit 
\begin{align}
    \theta &\approx \frac{L\omega}{2cn}\chi_{\text{xy}} \label{eq:theta_rl_chi} \\
    &= \frac{e_0^3\omega^2 NBL} {2\varepsilon_0 cm^2\left(\left(\omega_0^2-\omega^2\right)^2-\left(\omega \cdot \omega_\text{c}\right)^2\right)n},
\end{align}
wobei $\omega_0 = \sqrt{\frac{K}{m}}$ die Resonanzfrequenz, $\omega_\text{c} = \frac{e_0B}{m}$ als Zyklotronfrequen und $m$ die Masse ist. Die Mess- und Resonanzfrequenz
liegt bei Galliumarsenid im Infrarotbereich, wodurch $\left(\omega_0^2 - \omega^2\right)^2 \gg \omega^2\omega_c^2$ gilt, kann der Drehwinkel in Abhängigkeit der Wellenlänge
$\lambda$ der Welle weiter genähert werden:
\begin{equation}
    \theta \approx \frac{e_0^3 NBL 2 \pi^2 c}{\varepsilon_0 m^2 \lambda^2 \omega_0^4 n}.
\end{equation}
Für $\omega_0 \rightarrow 0$ und der effektiven Masse folgt schließlich für die Rotation der Polarisationsebene pro Einheitslänge
\begin{equation}
    \theta_\text{frei} \approx \frac{e_0^3 N B}{8\symup{\pi}^2 \varepsilon_0 c^3 (m^*)^2 n} \cdot \lambda^2.
\end{equation}
