\chapter{Diskussion}
\label{cha:Diskussion}
Als erstes wurde die maximale Feldstärke bestimmt. Diese ergab sich zu $B_\mathrm{max} = \qty{430}{\milli\tesla}$. Abbildung \ref{fig:magnetfeld} zeigt keinen auffälligen Verlauf,
weshalb von einem stabilen Magnetfeld ausgegangen wird. 
Dann wurde die effektive Masse von Leitungselektronen bestimmt. Der Theoriewert lautet $m^* = \num{0.063}m_e \approx \qty{5.74e-32}{\kilo\gram}$. Von diesem Wert ergibt sich eine 
mittlere Abweichung um $\qty{40+-11}{\percent}$. Diese Abweichung ist relativ groß. Mögliche Gründe für diese Abweichung sind ,dass bei der Messung rein optisch ein Minimum 
festgestellt werden musste. Dazu kommt, dass dies nicht immer eindeutig möglich gewesen ist. Aus diesem Grund mussten auch die Messwerte der Wellenlängen $\qty{2,51}{\micro\metre}$
und  $\qty{2,65}{\micro\metre}$ aus der Rechnung rausgenommen werden. Bei diesen Messungen war es nur mit Raten möglich ein Minimum zu benennen. Weiter kann auch eine nicht perfekte 
Justage schnell zu großen Abweichungen führen. Ebenso deutet der stark vorhandende Achsenabschnitt in der Regression darauf hin, dass es in dem Aufbau systematische Unsicherheiten 
gibt. Dennoch kann der errechnete Wert den Theoriewert im Rahmen der Messunsicherheit bestätigen.