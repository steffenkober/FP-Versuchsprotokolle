\chapter{Auswertung}
\label{cha:Auswertung}
\section{Untesuchung des verwendeten Magnetfeldes}
\label{sec:magn}
Zunächst wurde das Magnetfeld des verwendeten Magneten untersucht. Der Fokus bei dieser Untersuchung liegt weniger aufm dem Verlauf des Magnetfeldes als auf dem Maximum. 
Aufgrund der Geometrie des Aufbaus liegt die Probe genau im Maximum des Feldes. 

Die gemessenen Feldstärken sind in Abbildung \ref{fig:magnetfeld} dargestellt. 

\begin{figure}
              \centering
              \includegraphics[width=.8\textwidth]{build/magnetfeld.pdf}
              \caption{Grafische Darstellung der Messdaten zur Bestimmung der magnetischen Kraftflussdichte.}
              \label{fig:magnetfeld}
\end{figure}
Das Maximum des Magnetfeldes liegt bei $B_\mathrm{max} = \qty{430}{\milli\tesla}$.

\section{Bestimmung der effektiven Masse der Leitungselektronen}
\label{sec:effektive_masse}
Zur Bestimmung der effektiven Masse der Leitungselektronen werden drei \ce{GaAs} Proben untersucht. Für unterschiedliche Ladungsträgerdichten wird zunächst eine hochreine \ce{GaAs}
Probe verwendet. Danach werden dann noch zwei unterschiedlich stark n-dotierte Proben verwendet. Eine Probe hat eine Dotierungsdichte von $N_{12} = \qty{1.2e18}{\per\cubic\centi\metre}$
mit einer Dicke entlang des Infrarot von $L_{12} = \qty{1.36}{\milli\metre}$ und wird im folgendem durch den Index $12$ referenziert. Die andere dotierte Probe hat eine 
Dotierungsdichte von $N_{28} = \qty{2.8e18}{\per\cubic\centi\metre}$ mit einer Dicke entlang des Infrarot von $L_{28} = \qty{1.296}{\milli\metre}$ und wird im folgendem durch den Index 
$28$ referenziert. Die hochreine \ce{GaAs} Probe hat eine Dicke von $L_\mathrm{rein} = \qty{1.296}{\milli\metre}$ und wird im folgendem durch den index \enquote{rein} referenziert.

Die Messwerte sind in den Tabellen \ref{tab:mw1}-\ref{tab:mw3} dargestellt. Aus diesen Messwerten wurde gemäß Gleichung \eqref{eqn:theta} der Faraday-Rotationswinkel errechnet und ebenfalls 
in den Tabellen \ref{tab:mw1}-\ref{tab:mw3} dargestellt. In den selben Tabellen befinden sich auch die Wellenlängen der neun unterschiedlichen Interferenzfilter, 
welche im Experiment verwendet wurden. 

\begin{table}
              \centering
              \caption{Messwerte zur undotierten Probe und nach \autoref{eqn:theta} bestimmter Drehwinkel der Faradayrotation.}
              \label{tab:mw1}
              \begin{tabular}{c c c c}
                \toprule
                $\lambda \mathbin{/} \unit{\micro\meter}$ & $\theta_{1} \mathbin{/} \unit{\degree}$ & $\theta_{2} \mathbin{/} \unit{\degree}$ &%
                 $\theta \mathbin{/} \unit{\degree}$ \\
                \midrule
                $1,06 $ & $4,36$ & $4,78$ & $0,21$ \\
                $1,29 $ & $4,71$ & $4,45$ & $0,13$ \\
                $1,45 $ & $4,49$ & $4,71$ & $0,11$ \\
                $1,72 $ & $4,63$ & $4,47$ & $0,08$ \\
                $1,96 $ & $4,42$ & $4,52$ & $0,05$ \\
                $2,156$ & $4,49$ & $4,38$ & $0,05$ \\
                $2,34 $ & $3,94$ & $4,01$ & $0,03$ \\
                $2,51 $ & $3,53$ & $3,46$ & $0,03$ \\
                $2,65 $ & $2,84$ & $2,93$ & $0,04$ \\
                \bottomrule
              \end{tabular}
\end{table}

\begin{table}
              \centering
              \caption{Messwerte zur Probe mit $N = \qty{1.2e18}{\centi\metre^{-3}}$ und nach \autoref{eqn:theta_diff} bestimmter Drehwinkel der Faradayrotation.}
              \label{tab:mw2}
              \begin{tabular}{c c c c}
                \toprule
                $\lambda \mathbin{/} \unit{\micro\meter}$ & $\theta_{1} \mathbin{/} \unit{\degree}$ & $\theta_{2} \mathbin{/} \unit{\degree}$ &%
                 $\theta \mathbin{/} \unit{\degree}$ \\
                \midrule
                $1,06 $ & $2,69$ & $2,53$ & $0,08$ \\
                $1,29 $ & $2,57$ & $2,71$ & $0,07$ \\
                $1,45 $ & $2,58$ & $2,67$ & $0,04$ \\
                $1,72 $ & $2,71$ & $2,60$ & $0,05$ \\
                $1,96 $ & $2,83$ & $2,72$ & $0,05$ \\
                $2,156$ & $2,74$ & $2,86$ & $0,06$ \\
                $2,34 $ & $3,30$ & $3,18$ & $0,06$ \\
                $2,51 $ & $3,54$ & $3,58$ & $0,02$ \\
                $2,65 $ & $4,29$ & $4,22$ & $0,03$ \\
                \bottomrule
              \end{tabular}
\end{table}

\begin{table}
              \centering
              \caption{Messwerte zur Probe mit $N = \qty{2.8e18}{\centi\metre^{-3}}$ und nach \autoref{eqn:theta_diff} bestimmter Drehwinkel der Faradayrotation.}
              \label{tab:mw3}
              \begin{tabular}{c c c c}
                \toprule
                $\lambda \mathbin{/} \unit{\micro\meter}$ & $\theta_{1} \mathbin{/} \unit{\degree}$ & $\theta_{2} \mathbin{/} \unit{\degree}$ &%
                 $\theta \mathbin{/} \unit{\degree}$ \\
                \midrule
                $1,06 $ & $4,47$ & $4,64$ & $0,09$ \\
                $1,29 $ & $4,64$ & $4,45$ & $0,10$ \\
                $1,45 $ & $4,47$ & $4,64$ & $0,09$ \\
                $1,72 $ & $4,59$ & $4,42$ & $0,09$ \\
                $1,96 $ & $4,31$ & $4,50$ & $0,10$ \\
                $2,156$ & $4,50$ & $4,28$ & $0,11$ \\
                $2,34 $ & $3,84$ & $4,08$ & $0,12$ \\
                $2,51 $ & $3,80$ & $3,63$ & $0,09$ \\
                $2,65 $ & $4,22$ & $4,43$ & $0,10$ \\
                \bottomrule
              \end{tabular}
\end{table}

Da der Faraday-Winkel sowohl von der Probendicke als auch von der Dotierungsdichte abhängt, muss die Faradayrotaion pro Einheitslänge errechnet werden. Dies geschieht durch 
\begin{equation}
              \Theta_\mathrm{frei} = \frac{\Theta}{L} - \frac{\Theta_\mathrm{rein}}{L_\mathrm{rein}} \,.
\end{equation} 

In dieser Gleichung is $\Theta$ ein gemesserner Winkel mit dotierter Probe der Dicke $L$.

Aus der theoretischen Überlegung zu $\Theta_\mathrm{frei}$ folgte Gleichung \eqref{eqn:t_frei}. Diese Gleichung legt Nahe, dass $\Theta_\mathrm{frei} \propto \lambda^2$ ist. 
Betracht man $\Theta_\mathrm{frei}$ als linear in $lambda^2$ kann die effektive Masse, welche in dem Proportionalitätsfaktor enthalten ist, bestimmt werden indem die Messwerte für 
$\Theta_\mathrm{frei}$ linear an $lambda^2$ fittet. Formel \eqref{eqn:t_frei} weißt außerdem keinen Konstanten Term auf. Allerdings wird für den Fit eine konstante Variable hinzugefügt
um statistische und systematische Unsicherheiten zu minimieren.

Die lineare Regression wird mittels scipy \cite{scipy} berechnet. Die Messwerte von $\Theta_\mathrm{frei}$ sind in Abbildung \ref{fig:fit} zusammen mit den berechneten 
Ausgleichsgeraden dargestelllt. 

\begin{figure}
              \centering
              \includegraphics{effektive_masse_fit.pdf}
              \caption{Werte der Faradayrotation pro Eineheitslänge $\theta_\text{frei}$ der dotierten Proben und ermittelte Ausgleichsgeraden.}
              \label{fig:fit}
\end{figure}

Die rot markierten Messdaten wurden aus der folgenden Berechnung rausgenommen. Es ergeben sich die Parameter der Ausgleichsgeraden 
\begin{align*}
              a_{12} &= \qty{5.074+-1.490e12}{\metre^{-3}} &  a_{28} &= \qty{12.47+-1.35e12}{\metre^{-3}} \\
              b_{12} &= \qty{9.403+-5.147}{\metre^{-1}} & b_{28} &= \qty{18.37+-4.65}{\metre^{-1}}.
\end{align*}

Wird $b$ vernachlässigt kann Gleichung \eqref{eqn:t_frei} durch den Ansatz der Regression nach 
\begin{equation}
              \label{eqn:t_final}
              m^* = \sqrt{\frac{e**3NB}{8\pi^2\epsilon_0c^3na}}
\end{equation}
umgesformt werden. Aus dieser Gleichung kann dann die effektive Masse bestimmt werden. Für die beiden dotierten Proben ergibe sich 
\begin{align*}
              m_1 &= \qty{8.1+-1.2e-32}{\kilo\gram} & m_2 &=  \qty{7.9+-0.4e-32}{\kilo\gram}.
\end{align*}
Dabei wurde das im Abschnitt \ref{sec:magn} bestimmte maximale Magentfeld verwendet und der Brechungsindex von \ce{GaAs} als $n = \num{3.354}$.
Der Mittwelwert lautet $\qty{8+-0.6e-32}{\kilo\gram}$.